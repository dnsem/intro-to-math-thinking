%%
%% Author: semenovd
%% 27/02/2019
%%

% Preamble
\documentclass[11pt]{article}

% Packages
\usepackage{amsmath}
\usepackage{upgreek}
\usepackage{amsthm}
\usepackage{amssymb}

% Document
\begin{document}

   \pagenumbering{gobble}
   \paragraph{Task}\mbox{} \\
      Prove that for any integer $n$, at least one of the integers $n, n + 2, n + 4$ is divisible by $3$.
   \paragraph{Solution}\mbox{} \\
   Task is equavalent to proving the following theorem:
   \newtheorem*{Theorem}{Theorem}
   \begin{Theorem}
       $\displaystyle \forall n \in \mathbb{N}: 3 \mid n \lor 3 \mid (n + 2) \lor 3 \mid (n + 4)$
   \end{Theorem}
   \begin{proof}
   By cases. \\
   By division theorem, all integers can be expressed in one of these forms: $3k, 3k + 1, 3k + 2$, for some $k \in \mathbb{Z}$. \\
   In case $n = 3k$, for some $k$, it is divisible by 3 (by definition of divisibility). \\
   In case $n = 3k + 1$, for some $k$, then $n + 2 = 3k + 3 = 3(k + 1)$ is divisible by 3 (by definition of divisibility). \\
   In case $n = 3k + 2$, for some $k$, then $n + 4 = 3k + 6 = 3(k + 2)$ is divisible by 3 (by definition of divisibility). \\
   Hence, we proved that for any integer $n$, at least one of the integers $n, n + 2, n + 4$ is divisible by $3$.
   \end{proof}

\end{document}