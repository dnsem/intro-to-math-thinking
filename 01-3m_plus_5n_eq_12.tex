%%
%% Author: semenovd
%% 27/02/2019
%%

% Preamble
\documentclass[11pt]{article}

% Packages
\usepackage{amsmath}
\usepackage{upgreek}
\usepackage{amsthm}
\usepackage{amssymb}

% Document
\begin{document}

   \pagenumbering{gobble}
   \paragraph{Task}\mbox{} \\
   Say whether the following is true or false and support your answer by a proof.
   \begin{equation*}
      \left( \exists m \in \mathbb{N} \right) \left( \exists n \in \mathbb{N} \right) \left( 3m + 5n = 12 \right)
   \end{equation*}
   \paragraph{Solution}\mbox{} \\
   Let's claim that $\left( \exists m \in \mathbb{N} \right) \left( \exists n \in \mathbb{N} \right) \left( 3m + 5n = 12 \right)$ is false and prove it.
   \newtheorem*{Theorem}{Theorem}
   \begin{Theorem}
       $\left( \forall m \in \mathbb{N} \right) \left( \forall n \in \mathbb{N} \right) \left( 3m + 5n \neq 12 \right)$
   \end{Theorem}
   \begin{proof}
   Since $12$ is even number, then either both $3m$ and $5n$ are even, or both are odd.
   $3m$ is even only if $m$ is even, similarly $5n$ is even only if $n$ is even.
   So, either both $m$ and $n$ are event, or both are odd. \\
   In case both $m$    and $n$ are event, here are the possibilities:
   \begin{align*}
      3 \cdot 2 + 5 \cdot 2 = 16
   \end{align*}
   Obviously, all other combinations of even $m$ and $n$ will yield even bigger numbers. \\
   In case both $n$    and $m$ are odd, here are the possibilities:
   \begin{align*}
      3 \cdot 1 + 5 \cdot 1 &= 8 \\
      3 \cdot 1 + 5 \cdot 3 &= 18 \\
      3 \cdot 3 + 5 \cdot 1 &= 14
   \end{align*}
   Obviously, all other combinations of odd $m$ and $n$ will yield even bigger numbers. \\
   Thus we tried all combinations of $m$ and $n$, and neither of them yielded $3m + 5n = 12$.
   Hence, $\left( \forall m \in \mathbb{N} \right) \left( \forall n \in \mathbb{N} \right) \left( 3m + 5n \neq 12 \right)$.
   \end{proof}


\end{document}