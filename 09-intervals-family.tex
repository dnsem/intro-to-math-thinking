%%
%% Author: semenovd
%% 27/02/2019
%%

% Preamble
\documentclass[11pt]{article}

% Packages
\usepackage{amsmath}
\usepackage{upgreek}
\usepackage{amsthm}
\usepackage{amssymb}

% Document
\begin{document}

   \pagenumbering{gobble}
   \paragraph{Task}\mbox{} \\
       Give an example of a family of intervals $A_{n}, n = 1,2,\mathellipsis$, such that $A_{n+1} \subset A_{n}$ for all n and
       $\displaystyle\bigcap_{n = 1}^{\infty}A_{n} = \{x|(\forall n)(x \in A_{n})\} = \emptyset$.
       Prove that your example has the stated property.
   \paragraph{Solution}\mbox{} \\
       Let's take $A_{n} = (0, \frac{1}{n})$

   \newtheorem*{Theorem}{Theorem}
   \begin{Theorem}
       Given a family of intervals $A_{n} = \left( 0, \frac{1}{n} \right), n \in \mathbb{N}$, for all $n \in \mathbb{N}$ $A_{n+1} \subset A_{n}$  and
       $\displaystyle\bigcap_{n = 1}^{\infty}A_{n} = \emptyset$.
   \end{Theorem}
   \begin{proof}
   First let's prove that for all $n \in \mathbb{N}$ $A_{n+1} \subset A_{n}$:\\
   Clearly, $\forall n \in \mathbb{N}: \frac{1}{n+1} < \frac{1}{n}$ and hence $\forall n \in \mathbb{N}: \left( 0, \frac{1}{n+1} \right) \subset \left(0, \frac{1}{n} \right)$.\\
   Now let's prove that $\displaystyle\bigcap_{n = 1}^{\infty}A_{n} = \{x|(\forall n)(x \in A_{n})\} = \emptyset$.\\
   Clearly, $0 \notin \displaystyle\bigcap_{n = 1}^{\infty} \left( 0, \frac{1}{n} \right)$ for all $n \in \mathbb{N}$.
   We will prove that $\forall p \in \mathbb{R} \left[ p \notin \displaystyle\bigcap_{n = 1}^{\infty}A_{n} \right]$, where $p$  is meant to be arbitrarily small number.\\
   Assume, on the contrary, $\exists p \in \mathbb{R} \left[ p \in \displaystyle\bigcap_{n = 1}^{\infty}A_{n}\right]$.\\
   We will show that there is an interval $A_{m} = \left( 0, \frac{1}{m} \right)$, for some $m \in \mathbb{N}$, such that $p \notin A_{m}$.\\
   Let's take $m$ such that $\frac{1}{m} < p$, i.e. $m > \frac{1}{p}$. We can always find such $m$ by Archimedean property of real numbers.
   Then $p$ is upper bound of $A_{m}$.
   Since $lub(A_{m}) = \frac{1}{m}$ and $\frac{1}{m} < p$ then $p \notin A_{m}$ (by definition of least upper bound). \\
   Hence, $\exists m \in \mathbb{N}: p \notin A_{m}$.
   But then it means that $p \notin \left( A_{m} \cap \displaystyle\bigcap_{n = 1, n \neq m}^{\infty}A_{n} \right)$ (by definition of intersection),
   which is the same as $p \notin \displaystyle\bigcap_{n = 1}^{\infty}A_{n}$.
   But it contradicts our assumption that $\exists p \in \mathbb{R} \left[ p \in \displaystyle\bigcap_{n = 1}^{\infty}A_{n} \right]$.\\
   Hence, $\displaystyle\bigcap_{n = 1}^{\infty}A_{n} = \emptyset$.
   \end{proof}

\end{document}