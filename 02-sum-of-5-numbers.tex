%%
%% Author: semenovd
%% 27/02/2019
%%

% Preamble
\documentclass[11pt]{article}

% Packages
\usepackage{amsmath}
\usepackage{upgreek}
\usepackage{amsthm}
\usepackage{amssymb}

% Document
\begin{document}

   \pagenumbering{gobble}
   \paragraph{Task}\mbox{} \\
   Say whether the following is true or false and support your answer by a proof: The sum of any five consecutive integers is divisible by 5 (without remainder).
   \paragraph{Solution}\mbox{} \\
   Task is equavalent to proving/disproving the following theorem:
   \newtheorem*{Theorem}{Theorem}
   \begin{Theorem}
       $\displaystyle \forall n \in \mathbb{Z}, n > 0: \left( 5 \mid \sum_{i = 0}^{4}\left(n + i\right) \right) \land \left( 5 \mid \sum_{i = 0}^{4}-(n + i) \right)$.
   \end{Theorem}
   \begin{proof}
   By induction. \\
   \emph{Initial step (n = 0)}: $\displaystyle \left( 5 \mid \sum_{i = 0}^{4}\left(0 + i\right) \right) \land \left( 5 \mid \sum_{i = 0}^{4}-\left(0 + i\right) \right)$ \\
   $\displaystyle = \left( 5 \mid 10 \right) \land \left( 5 \mid -10 \right)$ - this is true.\\
   \emph{Induction step}: Assume $\displaystyle \left( 5 \mid \sum_{i = 0}^{4}\left(n + i\right) \right) \land \left( 5 \mid \sum_{i = 0}^{4}-(n + i) \right)$. \\
   Then for $n + 1$:\\
   $\displaystyle \left( 5 \mid \sum_{i = 0}^{4}\left(\left(n + 1\right) + i\right) \right) \land \left( 5 \mid \sum_{i = 0}^{4}-\left(\left(n + 1\right) + i\right) \right)$ \\
   $\displaystyle = \left( 5 \mid \left( \sum_{i = 0}^{4}\left(n + i\right) +  \sum_{i = 0}^{4}1 \right) \right) \land \left( 5 \mid \left( \sum_{i = 0}^{4}-\left(n + i\right) + \sum_{i = 0}^{4}-1 \right) \right)$ \\
   $\displaystyle = \left( 5 \mid \left( \sum_{i = 0}^{4}\left(n + i\right) +  5 \right) \right) \land \left( 5 \mid \left( \sum_{i = 0}^{4}-\left(n + i\right) - 5 \right) \right)$. \\
   Since, by induction hypothesis, $\displaystyle 5 \mid \sum_{i = 0}^{4}\left(n + i\right)$, then $\displaystyle \sum_{i = 0}^{4}\left(n + i\right) = 5k $ for some $k \in \mathbb{Z}$.
   Similarly, $\displaystyle \sum_{i = 0}^{4}-\left(n + i\right) = 5l $ for some $l \in \mathbb{Z}$. \\
   Then we have
   $\displaystyle \left( 5 \mid \left( \sum_{i = 0}^{4}\left(n + i\right) +  5 \right) \right) \land \left( 5 \mid \left( \sum_{i = 0}^{4}-\left(n + i\right) - 5 \right) \right)$ \\
   $\displaystyle = \left( 5 \mid \left( 5k +  5 \right) \right) \land \left( 5 \mid \left( 5l - 5 \right) \right)$ \\
   $\displaystyle = \left( 5 \mid \left( 5(k +  1) \right) \right) \land \left( 5 \mid \left( 5(l - 1 \right) \right)$, which is obviously true. \\
   Thus, by principle of induction,\\
   $\displaystyle \forall n \in \mathbb{Z}, n > 0: \left( 5 \mid \sum_{i = 0}^{4}\left(n + i\right) \right) \land \left( 5 \mid \sum_{i = 0}^{4}-(n + i) \right)$, \\
   which consitutes the proof of original claim.
   \end{proof}

\end{document}