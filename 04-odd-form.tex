%%
%% Author: semenovd
%% 27/02/2019
%%

% Preamble
\documentclass[11pt]{article}

% Packages
\usepackage{amsmath}
\usepackage{upgreek}
\usepackage{amsthm}
\usepackage{amssymb}

% Document
\begin{document}

   \pagenumbering{gobble}
   \paragraph{Task}\mbox{} \\
      Prove that every odd natural number is of one of the forms $4n + 1$ or $4n + 3$, where $n$ is an integer.
   \paragraph{Solution}\mbox{} \\
   Task is equavalent to proving the following theorem:
   \newtheorem*{Theorem}{Theorem}
   \begin{Theorem}
       $\displaystyle \forall m \in \mathbb{N}: 2 \mid m \Rightarrow \exists n \in \mathbb{Z} \left[ m = 4n + 1 \lor m = 4n + 3\right]$.
   \end{Theorem}
   \begin{proof}
   Let's take arbitrary $m \in \mathbb{N}$.
   By division theorem, $m \in \mathbb{N}$ can be expressed in one of these forms: $m = 4n, m = 4m + 1, m = 4n + 2, m = 4m + 3$, for some $n \in \mathbb{Z}$. \\
   Of these forms, two express even numbers, $4n = 2 \cdot 2n$ and $4n + 2 = 2(2n + 1)$, since both are divisible by 2.\\
   Two remaining forms, $4n + 1$ and $4n + 3$, are not divisible by $2$, so they are odd.
   Since our choice of $m$ was arbitrary, this implies that if $m$ is odd, it can be expressed in one of the forms $4n + 1$ or $4n + 3$, for some $n \in \mathbb{Z}$.\\
   This concludes the proof.
   \end{proof}

\end{document}