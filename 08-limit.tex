%%
%% Author: semenovd
%% 27/02/2019
%%

% Preamble
\documentclass[11pt]{article}

% Packages
\usepackage{amsmath}
\usepackage{upgreek}
\usepackage{amsthm}
\usepackage{amssymb}

% Document
\begin{document}

   \pagenumbering{gobble}
   \paragraph{Task}\mbox{} \\
      Prove (from the definition of a limit of a sequence) that if the sequence $\left\{a_{n}\right\}_{n=1}^{\infty}$ tends to limit $L$ as $n \rightarrow \infty$,
      then for any fixed number $M > 0$,  the sequence $\left\{Ma_{n}\right\}_{n=1}^{\infty}$ tends to the limit $ML$.
   \paragraph{Solution}\mbox{} \\
   Task is equavalent to proving the following theorem:
   \newtheorem*{Theorem}{Theorem}
   \begin{Theorem}
       $\displaystyle \forall M \in \mathbb{R}, M > 0: \lim_{n \to \infty} \left\{a_{n}\right\}_{n=1}^{\infty} = L  \Rightarrow \lim_{n \to \infty} \left\{Ma_{n}\right\}_{n=1}^{\infty} = ML$.
   \end{Theorem}
   \begin{proof}
   Let's take arbitrary $M \in \mathbb{R}, M > 0$. By definition of limit,\\
   $\left(\forall \epsilon \in \mathbb{R}, \epsilon > 0 \right) \left(\exists n \in \mathbb{N}\right) \left(\forall m > n \right)\left[ \left|a_{n} - L\right| < \epsilon \right]$.\\
   Let's take arbitrary $\delta \in \mathbb{R}, \delta > 0$ and take $\epsilon = \frac{\delta}{M}$.
   Then, for this $\epsilon$, \\
   $\left(\exists n \in \mathbb{N}\right) \left(\forall m > n \right)\left[ \left|a_{n} - L\right| < \epsilon \right]$ \\
   $\Rightarrow$ $\left(\exists n \in \mathbb{N}\right) \left(\forall m > n \right)\left[M \left|a_{n} - L\right| < M\epsilon \right]$ (multiplying by $M > 0$)\\
   $\Rightarrow$ $\left(\exists n \in \mathbb{N}\right) \left(\forall m > n \right)\left[ \left|Ma_{n} - ML\right| < M\epsilon \right]$ \\
   $\Rightarrow$ $\left(\exists n \in \mathbb{N}\right) \left(\forall m > n \right)\left[ \left|Ma_{n} - ML\right| < M\left(\frac{\delta}{M}\right) \right]$ (by definition of $\epsilon$)\\
   $\Rightarrow$ $\left(\exists n \in \mathbb{N}\right) \left(\forall m > n \right)\left[ \left|Ma_{n} - ML\right| < \delta \right]$. \\
   Since $\delta$ was taken arbitrarily, $\left(\forall \delta \in \mathbb{R}, \delta > 0 \right) \left(\exists n \in \mathbb{N}\right) \left(\forall m > n \right)\left[ \left|Ma_{n} - ML\right| < \delta \right].$
   But this is precisely definition of $\displaystyle \lim_{n \to \infty} \left\{Ma_{n} \right\}_{n=1}^{\infty} = ML$.\\
   Hence, $\displaystyle \forall M \in \mathbb{R}, M > 0: \lim_{n \to \infty} \left\{a_{n}\right\}_{n=1}^{\infty} = L  \Rightarrow \lim_{n \to \infty} \left\{Ma_{n}\right\}_{n=1}^{\infty} = ML$.\\
   \end{proof}

\end{document}